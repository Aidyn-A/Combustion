\documentclass[12pt]{article}
\usepackage{graphicx}
\usepackage{amsmath}
\usepackage[margin=0.65in]{geometry}
\begin{document}
\title{LMC Code Changes Writeup}
\author{Will Pazner}
\maketitle

\section{LMC Code}
\subsection{Previous results}
Running a simple test using the unmodified \texttt{1D\_MISDC} code for a hydrogen flame 
results in the following results.\\

\begin{center}
\begin{tabular}{ c c c c c c }
$Y({\rm H}_2)$          & 2.89E-07 & 1.81 & 8.24E-08 & 1.66 & 2.61E-08 \\
$Y({\rm O}_2)$          & 8.55E-06 & 1.85 & 2.38E-06 & 1.74 & 7.10E-07 \\
$Y({\rm N}_2)$          & 1.07E-06 & 1.83 & 3.00E-07 & 1.72 & 9.11E-08 \\
$Y({\rm H}_2{\rm O})$   & 8.47E-06 & 1.85 & 2.34E-06 & 1.75 & 6.97E-07 \\
$Y({\rm H}_2{\rm O}_2)$ & 6.20E-08 & 2.25 & 1.30E-08 & 2.62 & 2.12E-09 \\
$Y({\rm HO}_2)$         & 5.03E-08 & 2.17 & 1.12E-08 & 2.28 & 2.31E-09 \\
$\rho$                  & 3.83E-08 & 1.84 & 1.07E-08 & 1.77 & 3.13E-09 \\
$T$                     & 8.83E-02 & 1.84 & 2.47E-02 & 1.74 & 7.42E-03 \\
$\rho h$                & 6.19E+01 & 1.82 & 1.75E+01 & 1.76 & 5.15E+00 \\
$U$                     & 1.65E-02 & 1.76 & 4.87E-03 & 1.43 & 1.81E-03 \\
\end{tabular}
\end{center}

\subsection{Piecewise linear advection term}
Instead of using the Godunov solver to find a time-centered velocity, we use 
the velocities at the endpoints. In 1-D this is particularly easy, because the 
velocity is entirely determined by the divergence condition
\begin{equation}
   \label{div-condn}
   \nabla \cdot U^n = \widehat{S}^n,
\end{equation}
which is accomplished through the \texttt{macproject} subroutine.

This is then used to compute the advection term $A^n$. Then, for every MISDC 
interation $k$, we can use the scalar quantities to compute 
$\widehat{S}^{n+1,(k)}$, which then determines the velocity $U^{n+1,(k)}$ by 
enforcing the same divergence condition (\ref{div-condn}). We can then compute 
the time-endpoint advection term $A^{n+1,(k)}$.

We store both $A^n$ as the variable \texttt{aofs\_old} and $A^{n+1,(k)}$ as 
\texttt{aofs\_new}. Then, updating the thermodynamic variables, we use the 
averaged quantity
\begin{equation}
   \label{adv-avg}
   \texttt{aofs}_\texttt{avg} := \frac{A^n + A^{n+1,(k)}}{2}
\end{equation}
instead of the time-centered value $A^{n+1/2}$ obtained using the Godunov solver.
Finally, when calling VODE, instead of using the piecewise constant source term 
$A^{n+1/2}$, we use the piecewise linear term given by $A^n$ and $A^{n,(k)}$ at 
the endpoints.

These changes give rise to the following results.\\

\begin{center}
\begin{tabular}{ c c c c c c }
$Y({\rm H}_2)$          & 2.71E-07 & 1.84 & 7.58E-08 & 1.67 & 2.38E-08 \\
$Y({\rm O}_2)$          & 8.25E-06 & 1.87 & 2.26E-06 & 1.76 & 6.68E-07 \\
$Y({\rm N}_2)$          & 9.89E-07 & 1.85 & 2.74E-07 & 1.72 & 8.30E-08 \\
$Y({\rm H}_2{\rm O})$   & 8.16E-06 & 1.87 & 2.23E-06 & 1.77 & 6.55E-07 \\
$Y({\rm H}_2{\rm O}_2)$ & 6.16E-08 & 2.25 & 1.29E-08 & 2.68 & 2.02E-09 \\
$Y({\rm HO}_2)$         & 5.04E-08 & 2.16 & 1.13E-08 & 2.25 & 2.36E-09 \\
$\rho$                  & 3.60E-08 & 1.86 & 9.91E-09 & 1.78 & 2.88E-09 \\
$T$                     & 8.49E-02 & 1.86 & 2.35E-02 & 1.76 & 6.95E-03 \\
$\rho h$                & 5.75E+01 & 1.83 & 1.62E+01 & 1.78 & 4.72E+00 \\
$U$                     & 1.98E-02 & 1.69 & 6.13E-03 & 1.45 & 2.24E-03 \\
\end{tabular}
\end{center}

When the source term is chosen to be a piecewise constant given by the average 
(\ref{adv-avg}) of the advection terms, very similar results are observed. \\

\begin{center}
\begin{tabular}{ c c c c c c }
$Y({\rm H}_2)$          & 2.71E-07 & 1.83 & 7.59E-08 & 1.65 & 2.41E-08 \\
$Y({\rm O}_2)$          & 8.24E-06 & 1.87 & 2.26E-06 & 1.75 & 6.74E-07 \\
$Y({\rm N}_2)$          & 9.89E-07 & 1.85 & 2.74E-07 & 1.71 & 8.38E-08 \\
$Y({\rm H}_2{\rm O})$   & 8.16E-06 & 1.87 & 2.23E-06 & 1.75 & 6.63E-07 \\
$Y({\rm H}_2{\rm O}_2)$ & 6.13E-08 & 2.26 & 1.28E-08 & 2.65 & 2.03E-09 \\
$Y({\rm HO}_2)$         & 5.02E-08 & 2.16 & 1.12E-08 & 2.27 & 2.33E-09 \\
$\rho$                  & 3.60E-08 & 1.86 & 9.93E-09 & 1.77 & 2.92E-09 \\
$T$                     & 8.48E-02 & 1.85 & 2.35E-02 & 1.74 & 7.04E-03 \\
$\rho h$                & 5.74E+01 & 1.83 & 1.62E+01 & 1.76 & 4.78E+00 \\
$U$                     & 1.94E-02 & 1.67 & 6.07E-03 & 1.43 & 2.25E-03 \\
\end{tabular}
\end{center}

\subsection{Piecwise linear diffusion term}
Now we will try to use piecewise linear term for diffusion. Making the required 
change in the code results in blow-up (i.e. instability). We see that for 256 
gridpoints, halving the time step remedies the instability, but for 512 
gridpoints, even dividing the time step by a quarter still results in 
instability.

As an attempt to understand the problem, I examined the results from the VODE 
integration more carefully. Subdiving the time-interval $[t_n, t_{n+1}]$ into a 
fixed number $N$ subintervals of size $\Delta t_{loc}$, I used VODE to 
successively solve the ODE on the intervals $[t_n + k\Delta t_{loc}, t_n + 
(k+1)\Delta t_{loc}]$, where $0 \leq k < N$. Examining the solution produce by 
VODE over the subintervals revealed no apparent instability or spurious 
osciallations.

In the hope that increasing the number of MISDC iterations would help reduce 
the splitting error, I ran the code for varying values of 
\texttt{misdc\_iterMAX}. I found that the more MISDC iterations, the faster the 
solution became unstable. In factor, when performing 15 MISDC iterations, the 
method cannot successfully complete even one time-step.

When I ran the code with only one MISDC iteration, instability was observed 
only after space-time refinement (i.e. with 2048 gridpoints).

Additionally, I tried doing one MISDC iteration with piecewise constants and 
one with piecewise linear. When the first iteration is constant, and the 
second linear, the method is unstable. When the first iteration is linear, and 
the second constant, the method appears to be stable.

\section{MISDC-style Backward Euler step}
Instead of solving the correction ODE using the VODE integrator, we instead 
solve the correction integral equation using a Backward Euler step. Results 
(using a grid of 256 points, and timestep $\Delta t = 0.1\times10^{-5}$, 
refining both by a factor of 2 each run) are:
\begin{center}
\begin{tabular}{ c c c c c c }
$Y({\rm H}_2)$          & 7.96E-08 & 2.02 & 1.96E-08 & 2.01 & 4.89E-09 \\
$Y({\rm O}_2)$          & 1.16E-06 & 1.99 & 2.91E-07 & 1.99 & 7.34E-08 \\
$Y({\rm N}_2)$          & 1.08E-07 & 1.98 & 2.76E-08 & 2.00 & 6.91E-09 \\
$Y({\rm H}_2{\rm O})$   & 1.18E-06 & 2.00 & 2.96E-07 & 1.99 & 7.44E-08 \\
$Y({\rm H}_2{\rm O}_2)$ & 4.27E-09 & 1.94 & 1.11E-09 & 1.97 & 2.83E-10 \\
$Y({\rm HO}_2)$         & 1.60E-08 & 1.98 & 4.05E-09 & 1.99 & 1.02E-09 \\
$\rho$                  & 1.04E-08 & 1.99 & 2.62E-09 & 1.99 & 6.62E-10 \\
$T$                     & 1.20E-02 & 2.00 & 3.01E-03 & 2.00 & 7.51E-04 \\
$\rho h$                & 1.00E+01 & 1.95 & 2.59E+00 & 1.99 & 6.51E-01 \\
$U$                     & 1.82E-02 & 2.09 & 4.30E-03 & 2.09 & 1.01E-03 \\
\end{tabular}
\end{center}

%\pagebreak
%\section{Toy Problem}
%We consider the simple linear ODE
%\begin{equation}
%    \label{lin}
%    y' = ay + dy + ry, y(0) = 1
%\end{equation}
%where $a$, $d$ and $y$ represent advection, diffusion, and reaction terms 
%respectively. The exact solution is given by $e^{(a+d+r)t}$.

%We want to solve this ODE using an MISDC-like iterative scheme. 
%As in the LMC code, we solve for advection explicitly (forward Euler) and 
%for diffusion implicitly (backwards Euler). Because of the structure of the 
%scheme, we never have to calculate the advection term explicity. Then, we 
%are left with a correction equation for reaction of the form
%\[
%    y^{(k+1)}(t) = y^n + \int_{t^n}^t \left(
%        A(y^{(k+1)}_A) - A(y^{(k)})
%      + D(y^{(k+1)}_{AD}) - D(y^{(k)})
%      + R(y^{(k+1)}) - R(y^{(k)})
%      \right) d\tau
%      + \int_{t^n}^t F(y^{(k)}) d\tau.
%\]
%This equation can be solved in a number of ways:
%\begin{itemize}
%    \item using a backward Euler step
%    \item differentiating and obtaining a differential equation, which is
%        then solved either exactly (using the analytical solution derived 
%        by hand or using Mathematica), with:
%        \begin{itemize}
%            \item constant forcing term
%            \item linear forcing term
%            \item higher degree forcing term (interpolating polynomial)
%        \end{itemize}
%\end{itemize}

%In all of the cases testing (making the equation stiffer by increasing the 
%magnitude of the $d$ and $r$ coefficients), all of these methods resulted 
%in the exepcted order of accuracy and stability.

%\subsection{Nonlinearity}
%In order to further test our method, we applied our numerical method to 
%the following equations (where $p(t) = cos(t)$, for example)
%\begin{gather}
%    \label{nonlin}
%    y'(t) = ay(t) + dy(t) + ry(t)(y(t) - 1)(y(t) - 1/2)\\
%    \label{lincos}
%    y'(t) = p'(t) + a(p(t)-y(t)) + d(p(t)-y(t)) + r(p(t) - y(t)) \\
%    \label{nonlincos}
%    y'(t) = p'(t) + a(p(t)-y(t)) + d(p(t)-y(t)) + ry^2(t)(p(t) - y(t)),
%\end{gather}
%where equation (\ref{nonlin}) is a nonlinear version of (\ref{lin}), and 
%equation (\ref{nonlincos}) is a nonlinear modification of (\ref{lincos}). The 
%exact solution to (\ref{lincos}) is given by $p(t)$. In both of these cases, 
%the method performed as expected, with the same accuracy and stability 
%properties as in the previous section. In contrast to the behavior observed 
%in the LMC code, there was no difference in performance or stability between 
%solving the reaction correction ODE using a piecewise constant or piecewise 
%linear forcing term.

%\subsection{PDE}
%To test a case more similar to the full LMC problem, we consider the PDE
%\begin{equation}
%    \label{pde}
%    y_t = ay_x + \epsilon y_{xx} + ry(y-1)(y-1/2).
%\end{equation}
%Our spatial domain is the interval $[0, 20]$.
%We enforce Dirichlet boundary conditions of $1$ on the left boundary, and 
%$0$ on the right boundary, and use the initial condition
%\[
%    y(x, 0) = y^0(x) = \frac{\tanh(10-2x)+1}{2}
%\]
%We use the method of lines to solve this PDE, discretizing the first 
%spatial derivative using the central difference operator, and discretizing 
%the Laplacian using a three-point stencil. We are then left with an ODE of the 
%form
%\[
%    y_t = A(y) + D(y) + R(y),
%\]
%where $A$ and $D$ are discretizations of the differential operators
%\[
%    a\frac{\partial}{\partial x},\qquad\epsilon \frac{\partial^2}{\partial x^2}
%\]
%respectively.

\end{document}
