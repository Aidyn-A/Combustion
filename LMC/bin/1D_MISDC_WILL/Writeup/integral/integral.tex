\documentclass[12pt]{article}
\usepackage{graphicx}
\usepackage{amsmath}
\usepackage{amssymb}
\usepackage{pifont}
\usepackage[margin=0.65in]{geometry}
\usepackage{enumerate}
\usepackage{caption}
\usepackage{float}
\usepackage{graphicx}
\usepackage{subfig}\usepackage{graphicx}
\usepackage{subfig}

\def\arraystretch{1.5}
\captionsetup{aboveskip=8pt}
\newcommand{\cmark}{\ding{51}}%
\newcommand{\xmark}{\ding{55}}%
\setlength{\parskip}{1em}

\begin{document}
\noindent
Equation (5) from my notes is
\[
\phi^{(k+1)}_t = R(\phi^{(k+1)},t) + D(\phi_{AD}^{(k+1),n+1})
                 + \frac{1}{2}\left(A(\phi^n) + D(\phi^n) + 
                    A(\phi^{(k+1),n+1})-D(\phi^{(k+1),n+1})\right).
\]
Integrating from $t^n$ to $t^{n+1}$, we have
\begin{multline*}
   \int_{t^n}^{t^{n+1}} \phi^{(k+1)}_t(\tau) d\tau = 
      \int_{t^n}^{t^{n+1}} R(\phi^{(k+1)},\tau) d\tau \\
      + \int_{t^n}^{t^{n+1}} \Big\{  D(\phi_{AD}^{(k+1),n+1})
                 + \frac{1}{2}\left(A(\phi^n) + D(\phi^n) + 
                    A(\phi^{(k+1),n+1})-D(\phi^{(k+1),n+1})\right) \Big\} d\tau
\end{multline*}
Here we are considering the case where the advection and diffusion terms are 
approximated as piecewise constants, so all the terms in the second integral 
on the right-hand side are actually constants, and can therefore be integrated 
exactly. The case for piecewise polynomials of any degree is the same.
Therefore,
\begin{multline*}
   \int_{t^n}^{t^{n+1}} R(\phi^{(k+1)},\tau) d\tau = \phi^{(k+1),n+1} - \phi^n \\
    - \Delta t \Big\{  D(\phi_{AD}^{(k+1),n+1}) 
                 + \frac{1}{2}\left(A(\phi^n) + D(\phi^n) + 
                    A(\phi^{(k+1),n+1})-D(\phi^{(k+1),n+1})\right) \Big\},
\end{multline*}
where the right-hand side conists entirely of known quantities.

\end{document}
