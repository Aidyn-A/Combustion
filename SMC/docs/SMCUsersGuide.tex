\documentclass[11pt,letterpaper]{article}

\usepackage[margin=0.8in]{geometry}

\usepackage{color}
\usepackage{amsmath}
\usepackage{amssymb}
\usepackage{fixmath}
\usepackage{hyperref}

\renewcommand{\vec}[1]{\mbox{\boldmath$#1$}}
\newcommand{\tensor}[1]{\mbox{\boldmath{\ensuremath{#1}}}}

\begin{document}

\title{SMC: A Compressible Navier-Stokes Code for Combustion}
\maketitle

\section{Introduction}

{\tt SMC} {\footnote{People often ask what SMC means.  The name was
    chosen by someone who was trained as an astrophysicist.  The
    center for Computational Sciences and Engineering (CCSE) has a low
    Mach number combustion code called LMC, which can also stand for
    Large Magellanic Cloud, a satellite galaxy of the Milky Way.  The
    LMC galaxy has a little brother called the Small Magellanic Cloud
    (SMC). The SMC code is also like a little brother of the LMC
    code.}}, developed at the Center for Computational Sciences and
Engineering (CCSE) at Lawrence Berkeley National Laboratory, is a
combustion code.  {\tt SMC} uses the {\tt BoxLib} library
(\url{https://ccse.lbl.gov/BoxLib/}), also developed at CCSE.  {\tt
  SMC} solves compressible Navier-Stokes equations for viscous
multi-component reacting flows.

\section{Basic Formulation and Algorithm}

The Navier-Stokes equations solved by {\tt SMC} are
\begin{align}
\frac{\partial \rho}{\partial t} + \nabla \cdot (\rho
    \vec{u})= { } & 0, \\
\frac{\partial \rho \vec{u}}{\partial t} + \nabla \cdot (\rho
    \vec{u}\vec{u}) + \nabla p= { } & \nabla \cdot
  \tensor{\tau}, \\
\frac{\partial \rho E}{\partial t} + \nabla \cdot [(\rho E + p)
  \vec{u}] = { } & \nabla \cdot (\lambda \nabla T) + \nabla \cdot
  (\tensor{\tau} \cdot \vec{u}) - \nabla \cdot [\sum_k\rho Y_k
  (\vec{V}_k -\vec{V}_c) h_k ], \\
\frac{\partial \rho Y_k}{\partial t} + \nabla \cdot (\rho Y_k \vec{u})
= { } & \rho \dot{\omega}_k  - \nabla \cdot [\rho Y_k
  (\vec{V}_k -\vec{V}_c) ], 
\end{align} 
where $\rho$ is the density, $\vec{u}$ is the velocity, $p$ is the
pressure, $E$ is the specific energy density (kinetic energy plus
internal energy), $k$ denotes species, $Y_k$ is the mass fraction of
species $k$, $\tensor{\tau}$ is the viscous stress tensor, $\lambda$
is the thermal conductivity, $T$ is the temperature, $h_k$ is
the enthalpy for species $k$, $\vec{V}_k$ is the species mass
diffusion velocity, $\vec{V}_c$ is the correction velocity for species
diffusion arisen in the mixture model, and $\dot{\omega}_k$ is the
production rate of species $k$.  The viscous stress tensor is given by
\begin{equation}
  \tau_{ij} = \eta \left(\frac{\partial u_i}{\partial x_j} +
    \frac{\partial u_j}{\partial x_i} - \frac{2}{3}
    \delta_{ij} \nabla \cdot \vec{u} \right) +
  \xi \delta_{ij} \nabla \cdot \vec{u},
\end{equation}
where $\eta$ is the shear viscosity and $\xi$ is the bulk viscosity.
A mixture model is employed in {\tt SMC}.  The species diffusion
velocity in the mixture model is given by
\begin{equation}
  \vec{V}_k = -D_{k} \nabla X_k - D_{k} (X_k-Y_k)
  \nabla (\ln{p}),
\end{equation}
where $D_k$ is the diffusion coefficient of species $k$ and $X_k$ is
the molar fraction of species $k$.  The correction velocity is given by
\begin{equation}
  \vec{V}_c = \sum_\ell Y_\ell \vec{V}_\ell.
\end{equation}

The {\tt SMC} algorithm is based on finite-difference methods.  For
first derivatives with respect to spatial coordinates, the standard
8th-order stencil is employed.  Using a new compact 8th-order stencil,
second derivatives are given by
\begin{equation}
\frac{\partial}{\partial x} \left(a(x) \frac{\partial u}{\partial x}
\right)\bigg{|}_{i} \approx \frac{H_{i+1/2} - H_{i-1/2}}{\Delta x^2}.
\end{equation}
Here
\begin{equation}
  H_{i+1/2} = \sum_{m=-3}^{4} \sum_{n=-3}^{4} a_{i+m} M_{mn} u_{i+n},
\end{equation}
where $M$ is a $8 \times 8$ matrix.  {\tt SMC} also has a {\tt S3D}
style algorithm for second derivatives, in which the first derivative
operator is applied twice to obtain a second derivative.

The system is evolved with either a 3rd-order Runge-Kutta method or a
spectral deferred correction (SDC) method.  {\tt SMC} uses
Navier-Stokes characteristic boundary conditions (NSCBC).  The cgs
units are used in {\tt SMC}.

\section{Getting Started}

{\tt SMC} requires the {\tt BoxLib} library to manage the grids,
parallelization, and compilation system.  It uses the {\tt
  BOXLIB\_HOME} environment variable to find the path to {\tt BoxLib}.
{\tt BoxLib} can be downloaded via {\tt git} as: \vspace{5pt}

\verb|git clone https://ccse.lbl.gov/pub/Downloads/BoxLib.git|\\

We will use the {\tt Combustion/SMC/bin/kerneltest} problem as an
example of running {\tt SMC}.  The {\tt kerneltest} directory contains
a file called {\tt GNUmakefile} that specify how to build the
executable.  Note GNU {\tt make} is required for building.  The {\tt
  GNUmakefile} has the following options:
\begin{itemize}
\item {\tt SMC\_MIN := t}\\
  Due to the complexity of SDC and NSCBC, the full version of {\tt
    SMC} is hard for code analysis.  Thus, we have created a
  minimalist version of {\tt SMC} that uses the Runge-Kutta method and
  assumes periodic boundaries in all directions .  This option
  determines whether we use the minimalist version or not.  Leaving
  this option empty will use the full version.
\item {\tt NDEBUG :=}\\
  This option determines whether we compile with support for debugging
  (usually also enabling some runtime checks). Setting NDEBUG := t
  turns off debugging.
\item {\tt MPI := t}\\
  This determines whether we are doing a parallel build, using the
  Message Passing Interface (MPI) library. Leaving this
  option empty will disable MPI. 
\item {\tt OMP :=}\\
  This determines whether we are using OpenMP to do parallelism within
  a shared memory node. OpenMP is used together with MPI, with MPI
  distributing the grids across the processors and within a
  shared-memory node, OpenMP allows many cores to operate on the same
  grid.  Leaving this option empty disables OpenMP.
\item {\tt COMP := gfortran}\\
  Set this to your Fortran compiler of choice.
\item {\tt MKVERBOSE := t}\\
  This determines the verbosity of the building process.
\end{itemize} 

The {\tt BoxLib} building system used by {\tt SMC} is located at {\tt
  \$\{BOXLIB\_HOME\}/Tools/F\_mk}.  The compiler flags are specified in {\tt
  \$\{BOXLIB\_HOME\}/Tools/F\_mk/comps/}.  If building with MPI, {\tt
  SMC} needs to know the location of MPI library.  You need to provide
the MPI locations in the {\tt F\_mk/GMakeMPI.mak} file.  Note that
some shells (e.g., {\tt tcsh}) use the {\tt \$HOST} environment
variable to store host name, whereas other shells (e.g., {\tt bash})
use {\tt \$HOSTNAME}.

To build the executable, type {\tt make} in the {\tt kernaltest/}
directory.  The executable will have a name like {\tt
  main.Linux.gfortran.debug.mpi.exe} depending on the compiler,
operating system, and some other options specified in the {\tt
  GNUmakefile}.  To run {\tt SMC} on a single processor, type
\vspace{5pt}

\verb|./main.Linux.gfortran.debug.mpi.exe inputs_SMC|\vspace{5pt}\\
in the {\tt kernaltest/} directory, where the {\tt inputs\_SMC} file
specifies runtime parameters. 

\section{Runtime parameters}

{\tt SMC} obtains runtime parameters from an inputs file.  We can also
override the value of any parameter by specifying it on the
command line as:\vspace{5pt}

\verb|./main.Linux.gfortran.debug.mpi.exe inputs_SMC --parameter value|\\

Below are a list of selected runtime parameters and their default values.
\begin{itemize}
\item {\tt verbose = 0}\\
  This determines verbosity.
\item {\tt stencil\_type = "compact"}\\
  This determines the stencil type for second derivatives. The options
  are {\tt "compact"} or {\tt "S3D"}.
\item {\tt stop\_time = -1.d0}\\
  Simulation stop time.
\item {\tt max\_step = 1}\\
  Maximum number of timesteps in the simulation.
\item {\tt restart = -1}\\
  This determines whether this is a restart run.  The value of -1
  means the simulation starts from scratch.  A non-negative value
  specifies which file to restart from.
\item {\tt plot\_int = 0}\\
  Number of timesteps between writing a plot file, if the value is
  positive. 
\item {\tt chk\_int = 0}\\
  Number of timesteps between writing a checkpoint file, if the value is
  positive.
\item {\tt cflfac = 0.5d0}\\
  CFL factor to use in the computation of the advection timestep
  constraint. 
\item {\tt init\_shrink = 1.d0}\\
  The multiplicative factor ($\le 1$) to reduce the initial timestep as 
  computed by the CFL constraint.
\item {\tt n\_cellx = -1}\\
  Number of cells in the $x$-direction.  It must be $\ge 4$.
\item {\tt n\_celly = -1}\\
  Number of cells in the $y$-direction.  It must be $\ge 4$.
\item {\tt n\_cellz = -1}\\
  Number of cells in the $z$-direction.  It must be $\ge 4$.
\item {\tt max\_grid\_size = 64}\\
  This determines the largest grid size of a box.  Note that each
  processor may contain multiple boxes.
\item {\tt prob\_lo\_x = 0.d0}\\
  Physical coordinates of lo-$x$ corner of problem domain
\item {\tt prob\_lo\_y = 0.d0}\\
  Physical coordinates of lo-$y$ corner of problem domain
\item {\tt prob\_lo\_z = 0.d0}\\
  Physical coordinates of lo-$z$ corner of problem domain
\item {\tt prob\_hi\_x = 1.d0}\\
  Physical coordinates of hi-$x$ corner of problem domain
\item {\tt prob\_hi\_y = 1.d0}\\
  Physical coordinates of hi-$y$ corner of problem domain
\item {\tt prob\_hi\_z = 1.d0}\\
  Physical coordinates of hi-$z$ corner of problem domain
\end{itemize}

\end{document}
